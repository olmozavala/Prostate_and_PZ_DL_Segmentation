\documentclass{article}
\usepackage{spconf,amsmath,graphicx}

\title{3D MULTISTREAM U-NET FOR PROSTATE AND PERIPHERAL ZONE SEGMENTATION}

\name
  {Olmo Zavala-Romero$^{\star}$
    \qquad Adrian L. Breto$^{\star}$
    \qquad Nicole Gautney$^{\star}$ 
    \qquad Yu-Cherng C. Channing$^{\star}$}
\secondlinename{
    Alan Dal Pra$^{\star}$
    \qquad Matthew C. Abramowitz$^{\star}$
    \qquad Alan Pollack$^{\star}$
    \qquad Radka Stoyanova$^{\star}$ }

\address
    {$^{\star}$University of Miami Miller School of Medicine, 
    Department of Radiation Oncology, \\ Miami, Florida, United States}
    

\begin{document}
\maketitle

\begin{abstract}
A modified version of the well established 3D U-net architecture is proposed
for automatic segmentation of the prostate and its peripheral zone (PZ) using
multiplanar MRI. The proposed architecture is a multistream convolutional
neuronal network that uses axial, coronal, and sagittal MRI series as input. 
428 scans from multiple vendors are used to test the robustness 
of the proposed architecture against distinct MRI magnets . Additionally, the 
effect of 3D data augmentation  is evaluated.
The proposed segmentation scheme achieved good results for prostate and PZ compared to expert
contoured volumes. Combining images from different MRI vendors on the training of 
the network is of paramount importance for training a universal model for
prostate and pheripheral zone segmentation. 
\end{abstract}

\begin{keywords}
Prostate cancer, multiparametric MRI, Deep Learning, U-Net, machine learning
\end{keywords}

\section{Introduction}
\label{sec:intro}
Accurate prostate segmentation on MRI datasets is required for many clinical and research 
applications. Furthermore, due to the different imaging properties of the peripheral (PZ) 
and transition zones (TZ) of the prostate, accurate zonal segmentation is also necessary. 
The prostate and zonal contours are necessary for computer aided diagnosis (CAD)
applications for staging, diagnosis, and treatment planning for prostate cancer. In 
a series of applications, prostate contours are fused with ultrasound images to
guide prostate biopsies. Automatic segmentation of the prostate, PZ and TZ on MR 
images provides an opportunity to broaden the current scope of research by facilitating 
studies that include large populations of subjects and/or studies that incorporate 
serial imaging of the prostate to grant a longitudinal picture of disease 
progression and response.  
Prostate MRI image segmentation has been an area of intense research \cite{litjens2014evaluation}. Earlier, the 
applied approaches varied from model-based \cite{chowdhury2012concurrent,toth2012multifeature},
 to atlas-based segmentation
\cite{4_klein2008automatic,5_cheng2014atlas, 6_xie2014low, 7_tian2015fully, 8_korsager2015use, 9_chilali2016gland}.
 Our group also evaluated the performance of atlas-based approach for prostate and prostate zones 
segmentation using data from different MR vendors and acquisition parameters\cite{10_padgett2018towards}. The advent 
of deep learning techniques, such as convolutional neural networks (CNN) has led to great success 
in image classifiation \cite{11_krizhevsky2012imagenet,12_simonyan2011immediate}. Recently, 
the U-Net architecture has been proposed \cite{13_ronneberger2015u} for medical imaging 
segmentation and has been applied to the prostate\cite{14_meyer2018automatic}.
In this work we present a modification of the U-Net architecture for segmentation of both 
the prostate and prostate zones. In continuation of our previous work for creating an 
universal segmentation tool, the network is evaluated for images from different MR vendors.  

\section{Methodology}
\label{sec:methods}
Met

\subsection{Preprocessing}
\label{subsec:prepro}
The preprocessing of the images include: automatic selection of a region
of interest, image resampling to a resolution of $0.5 x 0.5 x 0.5$ mm, contours interpolation
using optical flow, bias correction using the N4ITK algorithm \cite{n4itk}, and
image normalization to an interval of [0,1].

The selection of the region of interested was  

\section{Results}

Table \ref{tab:res_prost} shows the obtained DSCs for the segmentation of the prostate when the six trained models are used for the segmentation of the GE and Siemens dataset.  The displayed DSC is computed from the validation subset when the model is trained on the same dataset, and computed from the whole dataset when the model is trained from a different dataset. 
 \begin{table}[h]
    \label{tab:res_prost}
    \caption{Dice Similarity Coefficients between manual and CNN-generated prostate contours for GE and Siemens MRI vendors datasets. The trained models are divided in: with (w) data augmentation (DA) and without (wo) DA.}
    \begin{tabular}{lcc}
         \hline
          \textbf{Prostate Models} & \textbf{GE} & \textbf{Siemens }\\
         \hline
         %GE ROI/Original & $0.916\pm0.015$/$0.917\pm0.015$ & $0.465\pm0.198$/$0.466\pm0.198$ \\
         %GE & $\mathbf{0.917\pm0.015}$ & $0.466\pm0.198$ \\
         GE ROI/Original & $0.855\pm0.064$/$\mathbf{0.860\pm0.054}$ & $0.804\pm0.099$/$0.802\pm0.106$ \\
         \hline
         %Siemens ROI/Original & $0.261\pm0.119$/$0.276\pm0.130$ & $0.936\pm0.21$/$0.932\pm0.022$ \\
         %Siemens & $0.276\pm0.130$ & $\mathbf{0.932\pm0.022}$ \\
         Siemens ROI/Original & $0.262\pm0.118$/$0.288\pm0.139$ & $0.892\pm0.038$/$0.889\pm0.035$ \\
         \hline
         %Combined ROI/Original & $0.828\pm0.116$/$0.824\pm0.113$ & $0.909\pm0.032$/$0.907\pm0.031$\\
         %Combined & $0.824\pm0.113$ & $0.907\pm0.031$\\
         Combined ROI/Original & $0.830\pm0.112$/$0.827\pm0.109$ & $\mathbf{0.896\pm0.037}$/$0.892\pm0.036$\\
         \hline
    \end{tabular}
\end{table} 
When the model is trained with examples from one dataset and used to segment prostates from scans of the same MRI vendor the average DSCs are: 0.753 for GE and 0.893 for Siemens. When the datasets are combined during training, the average DSC are: 0.746 for GE and 0.909 for Siemens.  The results obtained for the Siemens dataset are comparable with current state of the art methods for prostate segmentation. When the model is trained with examples from one MRI vendor and then used to process images from a different vendor, the resulting DSCs are low (0.322 and 0.169).  %This result exhibit 
The above shows how sensible the model is to subtle changes in the training dataset. It also displays the importance of testing how well deep learning architectures generalize to other MRI vendors.  The use of data augmentation improves the generalization of the models in most cases but not with a clear tendency, further analysis is required.

Table \ref{tab:res_pz} shows the obtained DSCs for the segmentation of the PZ the six trained models.  The best DSCs (0.653 and 0.756) for segmenting the PZ of the prostate are obtained when the model is trained using the combined dataset.  
 \begin{table}[h]
    \label{tab:res_pz}
    \caption{Dice Similarity Coefficients (DSC) between manual and CNN-generated PZ contours for GE and Siemens.The trained models are divided in: with (w) data augmentation (DA) and without (wo) DA.}
    \begin{tabular}{lcc}
         \hline
          \textbf{PZ Models} & \textbf{GE Dataset} & \textbf{Siemens Dataset}\\
         \hline
         GE ROI/Original & $0.767\pm0.093$/$0.759\pm0.089$ & $0.537\pm0.204$/$0.539\pm0.204$ \\
         %GE  & $\mathbf{0.74\pm0.09}$ & $0.40\pm0.22$ \\
         \hline
         Siemens ROI/Original & $0.591\pm0.223$/$0.591\pm0.219$ & $0.808\pm0.085$/$0.808\pm0.087$ \\
         %Siemens & $0.58\pm0.21$ & $\mathbf{0.78\pm0.08}$ \\
         \hline
         Combined ROI/Original & $\mathbf{0.797\pm0.093}$/$0.788\pm0.093$ & $\mathbf{0.813\pm0.079}$/$0.811\pm0.79$\\
         %Combined & $0.75\pm0.10$ & $0.78\pm0.09$\\
         \hline
    \end{tabular}
\end{table}

The average DSCs of all the PZ models are lower than the coefficients for segmenting the prostate, which implies that PZ segmentation is a more challenging task.
Figure \ref{fig:resseg} shows one example of a prosate segmentation on the Siemens MRI vendor and a PZ segmentation on the GE MRI vendor. Both examples are from the middle layer of the prostate, and their corresponding DSC are 0.903 and 0.737.
 \begin{figure}[h]
    \centering
    \includegraphics[totalheight=.2\textheight]{figures/results/Prostate_Px_Challenge__P_yes_ROI_MIN_Case-0128.png}
    \includegraphics[totalheight=.2\textheight]{figures/results/Prostate_Px_Challenge__P_yes_ROI_MEAN_Case-0176.png}
    \includegraphics[totalheight=.2\textheight]{figures/results/Prostate_Px_Challenge__P_yes_ROI_MAX_Case-0337.png}
    \vspace{10mm}
    \includegraphics[totalheight=.2\textheight]{figures/results/Prostate_Px_Challenge__P_yes_Original_MIN_Case-0128.png}
    \includegraphics[totalheight=.2\textheight]{figures/results/Prostate_Px_Challenge__P_yes_Original_MEAN_Case-0085.png}
    \includegraphics[totalheight=.2\textheight]{figures/results/Prostate_Px_Challenge__P_yes_Original_MAX_Case-0016.png}
    \vspace{10mm}
    \includegraphics[totalheight=.2\textheight]{figures/results/Prostate_GE__GE_yes_ROI_MIN_Case-0518.png}
    \includegraphics[totalheight=.2\textheight]{figures/results/Prostate_GE__GE_yes_ROI_MEAN_Case-0544.png}
    \includegraphics[totalheight=.2\textheight]{figures/results/Prostate_GE__GE_yes_ROI_MAX_Case-0537.png}
    \vspace{10mm}
    \includegraphics[totalheight=.2\textheight]{figures/results/Prostate_GE__GE_yes_Original_MIN_Case-0518.png}
    \includegraphics[totalheight=.2\textheight]{figures/results/Prostate_GE__GE_yes_Original_MEAN_Case-0544.png}
    \includegraphics[totalheight=.2\textheight]{figures/results/Prostate_GE__GE_yes_Original_MAX_Case-0537.png}
    \label{fig:resseg}
    \caption{Prostate segmentations of Siemens (up) and GE (down) MRI vendors respectively. }
\end{figure} 

 \begin{figure}[h]
    \centering
    \includegraphics[totalheight=.2\textheight]{figures/results/PZ_Px_Challenge__P_yes_ROI_MIN_Case-0325.png}
    \includegraphics[totalheight=.2\textheight]{figures/results/PZ_Px_Challenge__P_yes_ROI_MEAN_Case-0319.png}
    \includegraphics[totalheight=.2\textheight]{figures/results/PZ_Px_Challenge__P_yes_ROI_MAX_Case-0026.png}
    \vspace{10mm}
    \includegraphics[totalheight=.2\textheight]{figures/results/PZ_Px_Challenge__P_yes_Original_MIN_Case-0325.png}
    \includegraphics[totalheight=.2\textheight]{figures/results/PZ_Px_Challenge__P_yes_Original_MEAN_Case-0319.png}
    \includegraphics[totalheight=.2\textheight]{figures/results/PZ_Px_Challenge__P_yes_Original_MAX_Case-0010.png}
    \vspace{10mm}
    \includegraphics[totalheight=.2\textheight]{figures/results/PZ_GE__GE_yes_ROI_MIN_Case-0481.png}
    \includegraphics[totalheight=.2\textheight]{figures/results/PZ_GE__GE_yes_ROI_MEAN_Case-0462.png}
    \includegraphics[totalheight=.2\textheight]{figures/results/PZ_GE__GE_yes_ROI_MAX_Case-0508.png}
    \vspace{10mm}
    \includegraphics[totalheight=.2\textheight]{figures/results/PZ_GE__GE_yes_Original_MIN_Case-0481.png}
    \includegraphics[totalheight=.2\textheight]{figures/results/PZ_GE__GE_yes_Original_MEAN_Case-0462.png}
    \includegraphics[totalheight=.2\textheight]{figures/results/PZ_GE__GE_yes_Original_MAX_Case-0508.png}
    \label{fig:ressegpz}
    \caption{PZ segmentations of Siemens (up) and GE (down) MRI vendors respectively. }
\end{figure} 

\section{Discussion/Conclusions}
\ref{sec:disc}
Manual contouring of the prostate requires slice-by-slice contouring 
in axial or in any of the other two views. Besides labor intensive, manual 
contouring is prone to inter-observer variability and imperfections 
in the 3D delamination (Figure 1). The U-net results in this paper show 
that in addition to efficiency and reproducibility, automatic prostate segmentation can be 
achieved with accuracy, similar to the expert’s manual contours reproducibility results. 


\bibliographystyle{IEEEbib}
\bibliography{theref.bib}

\end{document}
