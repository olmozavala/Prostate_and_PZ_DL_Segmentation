\section{Results}
Table \ref{tab:res_prost} shows the obtained DSCs for the segmentation of the prostate when the three trained models are used for segmenting the GE and the Siemens dataset. The displayed DSCs are computed from the validation set when the model is trained on the same MRI vendor, and are calculated from the whole dataset when the model is trained from data of a different MRI vendor.

% tab:res_prost
When the model is trained with examples from one dataset and used to segment prostates from scans of the same MRI vendor the average DSCs are $0.855$ for GE and $0.892$ for Siemens. When the datasets are combined during training, the average DSC are $0.830$ for GE and $0.896$ for Siemens.  The results obtained for the Siemens dataset are comparable with recent methods for prostate segmentation \cite{guo2016deformable, lozoya2018assessing, jia20183d}. When the model is trained with examples from one MRI vendor and then used to process images from a different vendor, the resulting DSCs are lower ($0.262$ and $0.804$). We attribute this large discrepancy in DSCs for cross training and testing, to the difference in variability within the GE and Siemens dataset. The Siemens dataset is more homogeneous than the GE, which makes its model less robust to changes but, at the same time, the Siemens model is better for segmenting images scanned from the same MRI machine. These results exhibit how sensible the model is for subtle changes in the training dataset and it also displays the importance of testing how well deep learning architectures in the medical field generalize to other MRI vendors. 

Figure \ref{fig:resseg} shows the middle axial slice for the lowest, closest to mean, and highest DSC obtained for prostate segmentation on the Siemens and GE dataset. In general, the DSC increases with respect to the volume size of the prostate, and predictions for the Siemens dataset follow better the contours from the experts. From these examples it is also clear the images obtained with the Siemens machine have better native resolution than the ones from GE, which may be another reason why the network performs better for this MRI vendor. 
% fig:resseg

Table \ref{tab:res_pz} shows the obtained DSCs for the segmentation of the PZ for the three trained models.  The best DSCs of $0.797$ and $0.813$ are obtained when the model is trained using the combined dataset.  

% tab:res_pz
The average DSCs of all the PZ models are lower than the coefficients for segmenting the prostate, which is expected as segmenting the PZ is a more challenging task. The obtained DSC for segmenting the PZ with the \emph{Combined} model (0.788 and 0.811) are better than what we found in the literature for similar databases (0.60, 0.68, 0.62, 0.75) \cite{mooij_automatic_2018,toth_simultaneous_2013, chilali_gland_2016, hutchison_pattern_2012}. Figure \ref{fig:ressegpz} shows the middle axial slice for the lowest, closest to mean, and highest DSC obtained for the segmentation of the PZ on the Siemens and GE dataset. 
% fig:ressegpz 