\section{Methodology}
\label{sec:methods}

\subsection{Dataset}
\label{subsec:dataset}
The prostate and the peripheral zone (PZ) were manually
contoured on T2-weighted Axial MRI (T2W) in MIM (MIM Software Inc,
Cleveland, OH, USA) by imaging experts (RS, ALB) with more than
25 years combined expertise in prostate cancer. The contours were cross
checked by the two imaging experts and reviewed by radiation
oncologists (MCA, AP) with extensive expertise in genitourinary
malignancies. Two datasets were considered: (i) imaging data from the
SPIE-AAPM-NCI PROSTATEx Challenge \cite{deukwoo_classification_2018} (328 cases), 
acquired on two different types of Siemens 3T MR scanners, the MAGNETOM Trio
and Skyra (Siemens, Erlangen, Germany), and  (ii) MRI data from
patients in radiotherapy clinical trials at the University of Miami (100 cases),
acquired on Discovery MR750 3T MRI (GE, Waukesha, WI, USA).



\subsection{Preprocessing}
\label{subsec:prepro}
The proposed preprocessing steps of the images consist of: 
bias correction using the N4ITK algorithm \cite{n4itk}, 
image normalization to an interval of [0,1],
automatic selection of a region of interest (ROI), 
image resampling to a resolution of $0.5 \times 0.5 \times 0.5$ mm, 
and contour interpolation using optical flow.


The selection of the ROI was proposed in \cite{anneke} and it 
consists of reducing the size of the three MRI series (axial, sagittal, and coronal)
by the intersection of its three corresponding rectangular prisms. Then resampling
of the images is performed using linear interpolation with the ITK software \cite{itk}. 

Manual contours made by the experts were carried on the original image
resolutions obtained from the MRI and not on the higher resoulution version; therefore, 
the necessity for interpolating them. 
The proposed interpolation is performed in two dimensions for every two consecutives slices
of the contour. First, optical flow is obtained between the two contours using the  
Farneback method. Then, the contours are interpolated linearly following
the obtained directions from the optical flow . Figure \ref{fig:of1} shows an
example of the resulting interpolated contour using the proposed method. 
\begin{figure}[h]
    \centering
    \includegraphics[totalheight=.15\textheight]{imgs/methodology/OF_1.png}
    \includegraphics[totalheight=.15\textheight]{imgs/methodology/OF_2.png}
    \caption{Contours provided by experts are interpolated using optical flow and
    linear interpolation. On the left:
    original contours with 17 slices. On the right: interpolated contours with 68 slices.}
    \label{fig:of1}
\end{figure}

\subsection{Proposed architecture}
The proposed CNN consist of a 3D multistream architecture that follows the analysis
and synthesis path of the 3D U-Net \cite{cciccek20163d}. The input of each stream is
the postprocessed ROI for one of three MRI scans (saggital, coronal, and axial), 
with a resolution of $168^3$. During the analysis phase, a combination
of two convolutional layers and one max pool layer is applied three times. The second 
convolutional layer in each set doubles the number of channels. 
In the synthesis phase a similar set of two convolutional layers and
one deconvolution is applied, followed by batch normalization.
Figure \ref{fig:nn} shows the proposed model, where 
all convolutional layers use filter size of $3 \times 3 \times 3$ and
rectified linear unit (ReLu) as the activation function; with the exception
of the last layer which uses  $1 \times 1 \times 1$ filter size and Sigmoid function. 
\begin{figure*}[h]
    \centering
    \includegraphics[totalheight=.3\textheight]{imgs/methodology/NN.png}
    \caption{Optical flow..}
    \label{fig:nn}
\end{figure*}

With the proposed architecture the training time is reduced in half, compared
to the original architecture of Meyer et al. \cite{anneke}. This
improve in performance is obtained by  reducing the number of filters 
from 192 to 128 in the last layer of the analysis path, and
the addition of batch normalization after every convolutional layer 
in the sythesis path.

\subsection{Training}
\label{subsec:training}
The selected optimization algorithm is Stochastic Gradient Descent (SGD) with a
learning rate $\alpha = 0.05$, momentum of 0.2 and decay of $10^{-7}$. The training is performed
for 1000 epochs, a batch size of 50, and an early stop mechanism for the validation
loss if not improved by $\delta = 0.001$ after 70 iterations. 

The loss function used for the training is the negative Dice similarity coefficient (DSC):
\begin{equation}
\text{Loss} = \frac{2 \sum_{i=1}^{N}p_it_i}{\sum_{i=1}^{N}p_i^2 + \sum_{i=1}^{N}t_i^2 + \varepsilon} 
\label{eq:dsc}
\end{equation}
%where N is the total number of voxels in the image, when training for prostate segmenation,
%and the number of voxels \textbf{inside} the prostate, when training for the PZ. The segmentation
%of the PZ assumes that we already know where the prostate is, so we do not take into
%account anything outside the prostate for the loss function. 
where N is the total number of voxels in the image, $p_i$ the voxel values for the 
prediction of the network, and $t_i$ the true voxel values of the prostate or PZ masks.

The dataset was split into 90\% for training and 10\% for
validation. Also, in order to compare the robustness of the models with respect to changes
in MRI vendor machines,  a distinct model was trained for the datasets GE (90
training and 10 validation), and Siemens (295 training and 33 validation).
The largest model is trained with both datasets (385 training and 43 validation). 

To analyze how 3D data augmentation can influence the sensitivity of the models
against different vendor machines, additional models were trained 
incorporating data augmentation. The data augmentation is performed by flipping the
images in the sagittal axis,  shifting in any direction up to 10\% of the image size, and
bluring the images using Gaussian blur up to $\sigma = 3$. Each data augmentation
method is applied with a random chance of $1/3$.

A total number of 12 models were trained from the combination of 
three datasets, using or not data agumentation, and training for the segmentation 
of the prostate or the PZ. 
