\section{Results}

Table \ref{tab:res_prost} has the obtained DSCs for the segmentation of the prostate
when the 6 trained models are used for the segmentation of the Ge and Siemens dataset. 
In the case the model is trained with the same data set, the displayed DSC is the one
obtained in the validation test. If the model is not trained with the same
dataset, then the DSC is the average value of the whole dataset. 
\begin{table}[h]
    \centering
    \begin{tabular}{|l|c|c|}
         \hline
          \textbf{Trained Width} &  GE & Siemens (n=33) \\
         \hline
         GE: wo/w DA & 0.748/0.758 & 0.305/0.34 \\
         \hline
         Siemens: wo/w DA & 0.150/0.310 & 0.898/0.889\\
         \hline
         GE \& Siemens: wo/w DA & 0.553/0.739 & 0.904/0.915\\
         \hline
    \end{tabular}
    \caption{Dice Similarity Coefficients (DSC) between manual and CNN-generated 
    prostate contours for GE and Siemens}
    \label{tab:res_prost}
\end{table}
When the model is trained with examples from one dataset and used to segment
prostates from scans of the same MRI vendor the DSCs are: 0.753 for GE and 0.893  for Siemens. 
The results obtained for the Siemens dataset are comparable with 
other groups REFS.  In the case the model is trained with examples from 
one dataset and used to segment scans from a diffent MRI vendor, then the
 average DSCs are low (0.322 and 0.169). This result exhibit how sensible the models
can be to subtle changes in the training dataset. It also displays the importance 
of testing how well the models generalize in other MRI vendors. 
The use of data augmentation improves the generalization of the models in most cases
but not with a clear tendency, further analysis is required. 

\begin{table}
    \centering
    \begin{tabular}{|l|c|c|}
        \hline
         \textbf{Trained Width} &  GE & Siemens (n=33) \\
         \hline
         GE: wo/w DA & 0.748/0.758 & 0.305/0.034 \\
         \hline
         Siemens: wo/w DA & 0.150/0.310 & 0.898/0.889\\
         \hline
         GE \& Siemens: wo/w DA & 0.553/0.739 & 0.904/0.915\\
         \hline
    \end{tabular}
    \caption{Dice Similarity Coefficients (DSC) between manual and CNN-generated 
    PZ contours for GE and Siemens}
    \label{tab:res_pz}
\end{table}
