\section{Introduction}
\label{sec:intro}
Accurate prostate segmentation on MRI datasets is required for many clinical and research 
applications. Furthermore, due to the different imaging properties of the peripheral (PZ) 
and transition zones (TZ) of the prostate, accurate zonal segmentation is also necessary. 
The prostate and zonal contours are necessary for computer aided diagnosis (CAD)
applications for staging, diagnosis, and treatment planning for prostate cancer. In 
a series of applications, prostate contours are fused with ultrasound images to
guide prostate biopsies. Automatic segmentation of the prostate, PZ and TZ on MR 
images provides an opportunity to broaden the current scope of research by facilitating 
studies that include large populations of subjects and/or studies that incorporate 
serial imaging of the prostate to grant a longitudinal picture of disease 
progression and response.  
Prostate MRI image segmentation has been an area of intense research \cite{litjens2014evaluation}. Earlier, the 
applied approaches varied from model-based \cite{chowdhury2012concurrent,toth2012multifeature},
 to atlas-based segmentation
%\cite{4_klein2008automatic,5_cheng2014atlas, 6_xie2014low, 7_tian2015fully, 8_korsager2015use, 9_chilali2016gland}.
\cite{4_klein2008automatic, 8_korsager2015use, 9_chilali2016gland}.
 Our group also evaluated the performance of atlas-based approach for prostate and prostate zones 
segmentation using data from different MR vendors and acquisition parameters\cite{10_padgett2018towards}. The advent 
of deep learning techniques, such as convolutional neural networks (CNN) has led to great success 
in image classifiation \cite{11_krizhevsky2012imagenet,12_simonyan2011immediate}. Recently, 
the U-Net architecture has been proposed \cite{13_ronneberger2015u} for medical imaging 
segmentation and has been applied to the prostate\cite{14_meyer2018automatic}.
In this work we present a modification of the U-Net architecture for segmentation of both 
the prostate and prostate zones. In continuation of our previous work for creating an 
universal segmentation tool, the network is evaluated for images from different MR vendors.  
