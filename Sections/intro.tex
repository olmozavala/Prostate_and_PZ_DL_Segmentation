\section{Introduction}
\label{sec:intro}
itAccurate prostate segmentation on MRI datasets is required for many clinical and research applications. In addition, due to the different imaging properties of the peripheral (PZ) and transition zones (TZ) of the prostate, accurate zonal segmentation is also necessary. The prostate and zonal contours are required for computer-aided diagnosis (CAD) in applications for staging, diagnosis, and treatment planning for prostate cancer. In a series of applications, prostate contours are fused with ultrasound images to guide prostate biopsies. Automatic segmentation of the prostate, PZ and TZ on MR images provides an opportunity to broaden the current scope of research by facilitating studies that include large populations of subjects or studies that incorporate serial imaging of the prostate to grant a longitudinal picture of disease progression and response.  Prostate MRI image segmentation has been an area of intense research \cite{litjens2014evaluation}. Earlier, the applied approaches varied from model-based \cite{chowdhury2012concurrent,toth2012multifeature}, to atlas-based segmentation \cite{4_klein2008automatic,5_cheng2014atlas,6_xie2014low,7_tian2015fully,8_korsager2015use,9_chilali2016gland}.  Our group also evaluated the performance of atlas-based approach for prostate and prostate zones segmentation using data from different MRI vendors and acquisition parameters \cite{10_padgett2018towards}. The advent of deep learning techniques, such as convolutional neural networks (CNN) has led to great success in image classification \cite{11_krizhevsky2012imagenet,12_simonyan2011immediate}.  The winners of the PROMISE12 MICCAI Grand Challenge  for the automatic segmentation of the prostate,\cite{litjens2014evaluation}  used a volumetric Convolutional Neural Network (CNN) and achieved a Dice Coefficient (DSC) of 89.43\% \cite{yu2017volumetric}.  Recently, the U-Net architecture has been proposed \cite{13_ronneberger2015u} for medical imaging segmentation and multiple variations have been successfully applied to the prostate  \cite{anneke}. 
In this work, we augment upon the architecture of the 3D
CNN and analyze its performance for the automatic segmentation of the  prostate and PZ. As with our atlas-based approach \cite{10_padgett2018towards}, the goal of the described developments was to segue towards a universal approach that can segment the prostate and prostate zones, regardless of acquisition protocols, magnetic field strength or type of scanners. A core contribution is made by the pre-processing of the images in order to harmonize the data and optimizing the network performance when training and testing is carried out in a image dataset from two different MRI-vendors. 
